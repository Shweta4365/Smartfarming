\documentclass[12pt]{article}
\usepackage{graphicx}

\begin{document}
\title{Smart Farming}	
\maketitle

 \textbf{Introduction/Motivation:} 
\\* Smart farming is the practice of modern technologies such as sensors, robotics, and data analysis for shifting from tedious operations to continuously automated processes. 
Smart farms are expected to produce more yields with higher productivity at lower
expenses in a sustainable way that is less dependent on the manual force. Agricultural scientists, farmers, and growers are facing the challenge of producing more food from less land in a sustainable way to meet the demands of the predicted 9.8 billion populations in 2050. That is the equivalent of feeding a newly added city of 200 000 people every day. Combination of digital tools, sensors, control technologies all together have accelerated design and developments of agricultural robotics, demonstrating significant potentials and benefits in smart farming. Autonomous farm machinery equipped with local and global sensors for operating in row-crops have already become mature. Agricultural field robots  have become an important part in different aspects of smart farming. Agricultural field robots contributes to increasing the reliability of operations,  and improved yield with less human intervention. \\*
Keywords – agricultural robotics, precision agriculture, smart agriculture.\\*


\textbf{Market Research / Literature Survey:} \\*

\textbf{Smart Farming over Normal Farming:} \\*
Due to implementation of smart farming techniques these chances can be reduced to a great extent. Farmers can make crop selection effectively as initial stage of agriculture. Ground robotics system reduce the labour force and also the efficiency of doing particular task increases. \\*

\textbf{Declining Agri contribution to GDP:} \\*
The contribution of agriculture to the GVA has decreased from 15\% in 2015-16 to 14.4\% in 2018-19. The decline was mainly due to decrease in the share of GVA of crops from 9.2\% in 2015-16 to 8.7\% in 2017-18. \\*

\textbf{Profit to Government :} \\*
The total amount of subsidy provided for above equipments is Rs. 3,68,000, irrespective of other subsidies. Thus for particular agricultural field Indian Government provides approximately 5-6 lakhs including all the expenses. Propsed product can reduce this figures significantly. \\*


\textbf{Hardware requirements:}
\begin{itemize}
\item[$\cdot$] Plough
\item[$\cdot$] Drill Bits and Shafts
\item[$\cdot$] Track Belts, Wheels, Servo Horns
\item[$\cdot$] Aluminium Chassis
\item[$\cdot$] Funnels
\item[$\cdot$] Cutter Blade
\item[$\cdot$] 10 Litres Water storage Tank (Aluminium OR 3D Printed)
\item[$\cdot$] PVC Pipes
\item[$\cdot$] Vertical Sprayers
\item[$\cdot$] Conveyer Belt
\item[$\cdot$] 3D Printed Parts (Including Casing For Each System)
\end{itemize}

\textbf{Software requirements:}
\begin{itemize}
\item[$\cdot$] Fusion 360
\item[$\cdot$] APM Planner
\item[$\cdot$] Mission Planner
\item[$\cdot$] Flash print
\item[$\cdot$] Arduino
\end{itemize}

\textbf{Implementation:} \\*
Farmers will 
We have segregated the proposed design into 3 subdivisions as follows:
\begin{itemize}
\item[$\cdot$] \textbf{IOT BASED FARM MONITORING :} 
\end{itemize}
By using various smart agricultural gadgets, farmers will gain better control over the process of raising livestock and growing crops, making it more predictable and efficient. We will explore the use of IoT in agriculture and examine its benefits. We have used IoT for several processes required in efficient agriculture. The initial procedure of crop selection and crop rotation can be done by monitoring climate changes like rainfall patterns, humidity levels and incidence of sunlight by using several sensors. Also we can monitor changes in soil quality which depends on pH level, soil moisture and nutrient content. Crop monitoring is the after process for which surveillance is required. Thus our designed application allows a farmer to predict optimal weather conditions at sowing time under the availability of local networks.     

\begin{itemize}
	\item[$\cdot$] \textbf{GROUND ROBOTICS SYSTEM :}
\end{itemize}
 There are 8 major steps in agriculture. We will employ a single robot for minimum use of power to implement 3 major processes in growing of crop. The process of breaking the ground also known as tilling will be done by installing cultivator shanks beneath the robot for turning of soil and breaking of soil. A seeder with multiple centrifugal spreader is to be placed below at the center of the base. The seeder will place the seeds in the growth by pressing the base sheet towards the ground at least 3-4 inches deep. A tin sheet can be used to cover seeds with soil for proper cultivation.Second robot will be used for watering and harvesting of the crops. It will have a removable cutter at the front which will be used for cutting and a conveyor belt which will put the harvest in storage tank which  will also be used for storage of water.  

\begin{itemize}
	\item[$\cdot$] \textbf{ AERIAL SPRAYING SYSTEM :}
\end{itemize}
This system will have a drone which will have sprinklers attached to respective arms. Drones can spray the correct amount of liquid, modulating distance from the ground and spraying in real time for even coverage. The result will be increased efficiency with a reduction in the amount of chemicals penetrating into groundwater. This dramatically reduces the amount of chemicals used and virtually eliminates overspray. In fact, many experts estimate that aerial spraying can be completed up to five times faster with aerial drones than with traditional machinery. \\*



\textbf{User Interface :}
\\* Farmers will be provided user friendly software which will first help them for selection of crops as per the climate change. Also it will monitor the soil quality, soil moisture and nutrient content. 
There will be one drone and two robots on the land. The land robot will carry seeds and will drop seeds to the farmland. Same robot will do the work of ploughing, drilling. Another robot complete the task of watering and harvesting. At the same time user will send aerial drone for spraying the pesticides \\*


\textbf{Feasibility:}
\\* Indian Government provides subsidies to the farmers in each step of agriculture.According to the report by Sub Mission on Agricultaral Mechanization (SMAM) 12th plan period , assistance provided for few equipments  is as follows : \\*

\begin{tabular}{||l|c|r||}
	\hline
	Equipments & Subsidies provided & Assistance \\
	\hline
	Tractor & 1.00 lakh & 35\% \\
	Tiller & 0.50 lakh & 50\% \\
	Digger and other & 0.63 lakh & 50\% \\
	Chisel Plough & 0.08 lakh & 50\% \\
	Seed drill and other & 0.015 lakh & 50\% \\
	Seeder & 0.44 lakh &  50\% \\
	Flail Harvester & 0.25 lakh & 50\% \\	
	Electronic Sprayer & 0.63 lakh & 50\% \\
	\hline
\end{tabular} \\* \\*

 So if we calculate this total value the farmers will be able to get assistance of approx 4 lakh. Our overall system when launched as a product will cost less than 1.5 lakh in which we re providing all the above functionalities at considerably less cost.So it will be feasible for implementation.as our current project is a prototype it is smaller than the actual product at a product level it would be large enough to seed,sow,water and harvest two line sections of crops.  \\*
 
 
\begin{figure}
	\centering
	\includegraphics{bdsf1.png}
	\caption{Block Diagram 1}
	
	\includegraphics{bdsf2.png}
	\caption{Block Diagram 2}
	
	\label{image_2} % Label is used for referencing
	\label{image_1} % Label is used for referencing
\end{figure}


\begin{figure}
	\centering
	\includegraphics{sf1.png}
	\caption{ Diagram 1} 
	
	\includegraphics{sf2.png}
	\caption{ Diagram 2} 
	
	
	\label{image_1} % Label is used for referencing
	\label{image_2} % Label is used for referencing
	
\end{figure}

\begin{figure}
	\centering
	\includegraphics{sf3.png}
	\caption{ Diagram 3} 
	
	\includegraphics{sf4.png}
	\caption{ Diagram 4} 
	
	\label{image_3} % Label is used for referencing
	\label{image_4} % Label is used for referencing
	
\end{figure}

\begin{figure}
	\centering
	\includegraphics{sf5.png}
	\caption{ Diagram 5} 
	
	\includegraphics{sf6.png}
	\caption{ Diagram 6} 
	\label{image_5} % Label is used for referencing
	\label{image_6} % Label is used for referencing
\end{figure}

\begin{figure}
	\centering
	\includegraphics{sf7.png}
	\caption{ Diagram 7} 
	
	\includegraphics{sf8.png}
	\caption{ Diagram 8} 
	\label{image_7} % Label is used for referencing
	\label{image_8} % Label is used for referencing
\end{figure}



\textbf{Refrences:}\\*
https://farmer.gov.in \\*
https://www.agrimoon.com \\*
https://www.agrifarming.in/smart-farming-in-india-challanges-techniques-benefits

\end{document}